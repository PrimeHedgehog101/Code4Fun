\documentclass[1pt]{article}
\usepackage{amsmath}
\usepackage{amssymb}

% Bracketing

\newcommand{\brac}[1]{\left(#1\right)}
\newcommand{\sbrac}[1]{\left[#1\right]}
\newcommand{\set}[1]{\left\{#1\right\}}
\newcommand{\abs}[1]{\left|#1\right|}

% Abbreviations
\newcommand{\bb}[1]{\mathbb{#1}}
\newcommand{\mc}[1]{\mathcal{#1}}
\newcommand{\eps}{\varepsilon}
\newcommand{\es}{\varnothing}
\newcommand{\ra}{\rightarrow}
\renewcommand{\d}{\mbox{d}}
\newcommand{\e}{\mbox{e}}

% Calculus
\newcommand{\dx}{\; \mbox{d}x}

\renewcommand{\vec}[1]{\boldsymbol{\underline{#1}}}
\newcommand{\uvec}[1]{\boldsymbol{\underline{\hat{#1}}}}
\newcommand{\pd}[3]{\dfrac{\partial^{#3} #1}{\partial {#2}^{#3}}}
\newcommand{\diff}[2]{\dfrac{\text{d} #1}{\text{d} #2}}
\newcommand{\ddiff}[2]{\dfrac{\text{d}^2 #1}{\text{d} {#2}^2}}
\newcommand{\prt}[1]{\dfrac{\partial}{\partial #1}}

% Linear Algebra

\newcommand{\mat}[4]{
   \begin{pmatrix}
      {#1} & {#2} \\
      {#3} & {#4}
   \end{pmatrix}
   }

\newcommand{\cvec}[3]{
   \begin{pmatrix}
      {#1} \\
      {#2} \\
      {#3}
   \end{pmatrix}
}

\newcommand{\ccvec}[2]{
   \begin{pmatrix}
      {#1} \\
      {#2} 
   \end{pmatrix}
}

\usepackage{hyperref}

\usepackage[left=1cm, right=1cm, top=1cm,bottom=2cm,]
{geometry}


\begin{document}

\section{2021-22 Question 1}

In a simple one-dimensional continuum model for traffic flow, the density
of cars at point x and time t is given by $\rho(x, t)$, and can range between
0 and 9/4. The speed of cars depends only on the local density, and is
modelled as $v(\rho) = 1 - \frac{2}{3}\rho^{\frac{1}{2}}$. At $t = 0$ , the density of cars is given by
$$\rho(x, 0) = \rho_0(x) = \begin{cases}
0 & x\leq 0 \\
x^2 & 0<x<1 \\ 
1 & 1\leq x
\end{cases}$$
\subsection{ii) a)}
Write down the equation for conservation of cars. Show that the
characteristics in the $x-t$ plane are given by $\frac{dx}{dt}=1-\rho^{\frac{1}{2}}$ and explain why they must be straight lines.
\\
\\ The equation of conservation of cars is $$\frac{\partial \rho}{\partial t}+\frac{\partial}{\partial x}(\rho v)=0$$
\\ We are given that $v(\rho)=1-\frac{2}{3}\rho^{\frac{1}{2}}$, so we can write our governing equation as $$\frac{\partial \rho}{\partial t}+\frac{\partial}{\partial x}(\rho-\frac{2}{3}\rho^{\frac{3}{2}})=0$$ Using chain rule we have $$\frac{\partial \rho}{\partial t}+\frac{\partial}{\partial \rho}(\rho-\frac{2}{3}\rho^{\frac{3}{2}})\frac{\partial \rho}{\partial x}=0$$ hence we have $$\frac{\partial \rho}{\partial t}+\frac{\partial}{\partial \rho}(1-\rho^{\frac{1}{2}})\frac{\partial \rho}{\partial x}=0$$ We can now use characteristics to form the following three equations; $\frac{dt}{ds}=1$, $\frac{dx}{ds}=1-\rho^{\frac{1}{2}}$, $\frac{d\rho}{ds}=0$. Dividing 2 by 1 we get $$\frac{dx}{dt}=1-\rho^{\frac{1}{2}}$$
\\
\\ We will get straight lines for the gradient as $\rho$ is conserved along each characteristic. Since the gradient only depends on $\rho$ we have a constant gradient thus straight lines.
\subsection{b)}
Use the method of characteristics to solve for the density $\rho(x, t)$ for
$0\leq t\leq 1$. Show that a shock first forms at $x = 1 , t = 1$.
\\
\\ We have $$\frac{dx}{dt}=1-\rho^{\frac{1}{2}}$$ The gradient is constant, so this will be given by the boundary condition at $t=0$.
\\ Hence $$\frac{dx}{dt}=1-\rho_0(x_0)^{\frac{1}{2}}$$
\\ Integrating we get $$x=(1-p_0^{\frac{1}{2}})t+x_0$$ where $(x,t)=(x_0,0)$. 
\\ 
\\ We now need to evaluate this for the different values of $\rho_0$. These are given by $x_0\leq 0$, $0<x_0<1$ and $x_0\geq 1$. 
\\
\\ For $x_0\leq 0$ we have $\rho_0=0$ so $$x=t+x_0$$ The limiting characteristic (ie when the characteristics begin to change) is when $x_0=0$, so we have $$x=t$$ this is true for $x<t$.
\\
\\ For $0<x_0<1$, $\rho_0=x^2$ so $$x=(1-x_0)t+x_0$$ We already know what happens when $x_0=0$ as we just did that, so for $x_0=1$ the limiting characteristic is $$x=x_0=1$$ This is true for $t<x<1$. The limiting characteristic indicates a vertical line. 
\\
\\ For $x_0>1$, $\rho_0=1$ so $$x=x_0=1$$ for all $x_0>1$ so we have vertical lines. This is true for $x>1$.
\\
\\ A shock forms as when $x_0=0$ we have $x=t$ and when $x_0=1$ we have $x=1$. These meet at $t=1$ indicating a shock.
\subsection{c)}
Sketch $\rho$ as a function of x for some representative values of $t\leq 1$.
\\
\\ We know the density for $x\leq 0$ is $\rho_0=0$ and $x\geq 1$ is $\rho_0=1$ so we just need to work out the value of $\rho$ in-between. Hence for $0<x<1$ we have $\rho_0=x^2$ so $$x=(1-x_0)t+x_0$$ Rearranging for $x_0$ we have $x_0=\frac{x-t}{1-t}$. Substituting into $\rho_0=x^2$ we have $\rho(x,t)=\frac{(x-t)^2}{(1-t)^2}$ for $t<x<1$.
\\
\\Hence we have $$\rho(x,t)=\begin{cases}
0 & x\leq t \\
\frac{(x-t)^2}{(1-t)^2} & t<x<1 \\
1 & x\geq 1
\end{cases}$$







\end{document}
