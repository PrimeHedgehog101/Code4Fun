\documentclass[1pt]{article}
\usepackage{amsmath}
\usepackage{amssymb}


\usepackage[left=1cm, right=1cm, top=1cm,bottom=1cm,]
{geometry}


\begin{document}

\begin{center}
\section*{Complex integration/Rings Exam sheet}
\end{center}
\small
\subsection*{Complex Integration}

Circular curve: $a+bi+re^{it}$
\\Cauchy's intergral formulea: $i) \ \ f(a)=\frac{1}{2\pi i}\int_{\gamma}\frac{f(z)}{z-a}dz$ where $\gamma$ is closed, $f$ is holomorphic on and inside $\gamma^*$ and $a\in \gamma^*$.
\\ Cauchy s formula: $\int_{\gamma} f(z)dz=0$ if $f$ is holomorphic on and inside a simple closed path $\gamma$.
\\
\\ Definition 10.1 Let f be a complex function.
\\(i) The point $a \in \mathbb{C}$ is a regular point of f if f is holomorphic at a.
\\(ii) The point $a \in \mathbb{C}$ is a singularity of f if it is not a regular point but it is a limit
point of regular points. A singularity a of f is called
\\(a) an isolated singularity if f is holomorphic in some punctured disc $D'
(a; r)$;
\\(b) a non-isolated essential singularity otherwise.
\\
\\Residue Lemma 11.3: If $f$ has a simple pole at $a$ then $$res\{f;a\}=\lim_{z\rightarrow a}(z-a)f(z)$$

\subsection*{Rings}
Sub-ring test: For a sub set $S\subset R$, Sub ring if
\begin{itemize}
\item $0,1\in S$
\item $S$ is a subgroup of $(R,+)$: Show closure under addition and the additive inverse is in $S$.
\item Closure under multiplication.
\end{itemize} 
Unit: A unit $u$ if $\exists v \in R$ s.t $uv=1=vu$ (ie $u$ has a multiplicative inverse)
\\ $a,b$ are associates if $\exists$ a unit $u$ s.t $a=ub$.
\\ Given an associate, find all units, then multiply the associate with the units to find the other associates. (Find units by finding numbers co prime to mod n.











\newpage
\textbf{Open Set:} $S \subseteq \bb{C}$ is open if 
$$\brac{\forall z \in S}\brac{\exists \delta \in \bb{R}_{>0}} \; D\brac{z;\delta} \subseteq S.$$
\textbf{Closed Set:} $S \subseteq \bb{C}$ is closed if $\bb{C} \setminus S$ is open. \\[\baselineskip]
\textbf{Bounded Set:} $S \subseteq \bb{C}$ is bounded if 
$$\brac{\exists M \in \bb{R}} \; S \subseteq \overline{D}\brac{0,M} \text{ or equivalently } \brac{\forall z \in S} \; \abs{z} \leq M.$$
\\
\\
\textbf{Definition (Curve):} A curve is a continuous function $\gamma : \sbrac{a,b} \rightarrow \bb{C}$, where $\sbrac{a,b} \subseteq \bb{R}$.
We call $\gamma\brac{a}$ the initial point of $\gamma$ and $\gamma\brac{b}$ the final point of $\gamma$. \\[\baselineskip]

A curve $\gamma$ is said to be closed if $\gamma\brac{a} = \gamma\brac{b}$. We say that $\gamma$ is simple if it is injective on $\brac{a,b}$.
We say that $\gamma$ is smooth if it is differentiable and has a continuous derivative. \\[\baselineskip]

\textbf{Curve Inversion:} Given a curve $\gamma : \sbrac{a,b} \rightarrow \bb{C}$, the inversion of $\gamma$ is denoted by $-\gamma$ and 
it is defined by $-\gamma\brac{t} = \gamma\brac{b+a-t}$ for $t \in \sbrac{a,b}$. 
That is, $-\gamma$ is the same curve in the opposite direction. \\[\baselineskip]

\textbf{Joining Curves:} Let $\gamma_1 : \sbrac{a,b} \rightarrow \bb{C}$  and $\gamma_2 : \sbrac{b,c} \rightarrow \bb{C}$ such that
$\gamma_1\brac{b} = \gamma_2\brac{b}$. Then, 
\[
   \gamma_1 \cup \gamma_2 : \sbrac{a,c} \rightarrow \bb{C} = 
   \begin{cases}
      \gamma_1\brac{t}, & \text{if } t \in \sbrac{a,b}; \\
      \gamma_2\brac{t}, & \text{if } t \in \sbrac{b,c}.
   \end{cases}
\]

\textbf{Definition (Path):} A path is a curve which is the join of finitely many smooth curves. We call the image sets of paths shapes. \\[\baselineskip]

\textbf{Contour/Circline Path:} A circline path is a path which is a join of finitely many line segments and/or circular arcs. A contour
is a simple closed circline path. \\[\baselineskip]

\textbf{Region (Definition):} A set $S \subseteq \bb{C}$ is said to be path connected if for every $z,w \in S$ with $z \neq w$, 
there exists a path $\gamma$ such that the initial point of $\gamma$ is $z$, the final point is $w$, and the entire image $\gamma^*$ of $\gamma$
is contained in $S$. An open path connected set is called a region.
\\
\\ \textbf{Along a path:}
$$\int_{\gamma} f\brac{z} \; \d z := \int_{a}^{b} f\sbrac{\gamma\brac{t}}\gamma'\brac{t} \; \d t.$$
\textbf{Fundamental Estimate:} 
$$\abs{\int_{\gamma} f\brac{z} \; \d z} \leq \int_a^b \abs{f\sbrac{\gamma\brac{t}}} \cdot \abs{\gamma'\brac{t}} \; \d t.$$
\textbf{Corollary:} Put $M = \sup_{z \in \gamma^*}\abs{f\brac{z}}$ and $\ell = \int_a^b \abs{\gamma'\brac{t}} \; \d t$ (the length
of $\gamma$). Then,
$$\abs{\int_\gamma f\brac{z} \; \d z} \leq M\ell.$$







\end{document}
